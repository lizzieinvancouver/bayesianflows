\documentclass[11pt]{article}
\usepackage[top=1.00in, bottom=1.0in, left=1in, right=1in]{geometry}
\renewcommand{\baselinestretch}{1.1}
\usepackage{graphicx}
\usepackage{natbib}
\usepackage{amsmath}
\usepackage{parskip}
\usepackage{hyperref}

\def\labelitemi{--}

\usepackage{fancyhdr}
\pagestyle{fancy}
\fancyhead[LO]{}
\fancyhead[RO]{}

\begin{document}
\bibliographystyle{/Users/Lizzie/Documents/EndnoteRelated/Bibtex/styles/besjournals}
\renewcommand{\refname}{\CHead{}}

\title{How to fit Bayesian models and influence people \emph{or}\\
How to do Bayesian model fitting in ecology \emph{or}\\
The best way to be a Bayesian in ecology today}
\date{\today}
\maketitle

\section{Outline}

\begin{enumerate}
\item Intro 
\begin{enumerate}
\item The explosion of Bayesian approaches recently -- boom!
\item At the same time ecology has and is shifting (NCEAS etc.)
\begin{enumerate}
\item `The modern ecologist is computational.' 
\item Intro should include working across skillsets in different fields of ecology.
\end{enumerate}
\item Old Bayesian
\item So, where does that leave us?
\item Here we describe a broadly generalizable workflow for Bayesian analysis and show how it can revolutionize training in ecology by integrating more model building and model understanding. 
\item Will writes 3 paragraph explanation of Bayes (this goes in a box)
\end{enumerate}
\item General benefits---and pitfalls---of Bayesian 
\begin{enumerate}
\item You can fit any model you want! This is so great as it blurs the line between conceptual models, mechanistic models, process-based models, mathematical models (theory) and statistical models. Yay!
\item Because you can fit whatever you want, you can get numbers that really make sense -- numbers you may really want to estimate. This focus on effect sizes and what you want to estimate, moves away from p-values. Suddenly you can stop saying 'effect size' as some unitless Guretevitch value and say 'change in days per degree of warming under X model or such.' 
\item You can fit any model you want! So you can fit stuff that may be almost entirely uninfluenced by your data if you have no idea what you're doing. Oh dear.
\item So, how you do get all the benefits of Bayesian without doing something stupid?  
\end{enumerate}
\item Introducing the very basic workflow (just explain it Lizzie!) ... end or somewhere add: Many steps should be familiar to statistical ecologists, but are often overlooked. We provide further steps particular to Bayesian (prior predictive checks).
\item Benefits of this workflow
\begin{enumerate}
\item More fully integrates the mathematical model to statistical part of Bayesian -- you have to write the model with test data before you fit to your data
\item Both the test data writing and the posterior predictive checks dive you deep into understanding your mechanistic-statistical model and that, in our experience, gives you WAY more insights and ideas into your biological model and---wait for it---your biological system!
\end{enumerate}
\item Surprising things that happened to us that may happen to you if you follow this workflow
\begin{enumerate}
\item We got deeply in touch with the term {\bf nonidentifiability} and how it can happen (model nonidentifiability and data + model nonidentifiability) ... now that you can fit any model you want, you see this happen (before, a Hessian problem may have stood in your way sometimes for these models, but also sometimes for perfectly good models)
\item Simplify! Simplify! Simplify! Once you do the workflow, you may end up like us: fitting fewer levels in your mixed effects models, fitting fewer interactions.
\item Don't cram the world into your model ... in contrast, value of the posterior for manipulation. 
\item Plotting: The whole big wide world between plotting your raw data and plotting you model ... Plotting `partials' of your model (remove the site effects, for example).
\end{enumerate}
\item Challenges of the workflow compared to how we traditionally train ecologists (or, how this will reshape ecological training)
\begin{enumerate}
\item Good stats workflows bleed into what we expect of theoretical ecologists (and yet we act like non-theory folks should be trained differently). 
\item Theory = simpler models = outcome of a good Bayesian workflow (often)
\end{enumerate}
\item Moving away from old school Bayesian and into the light -- Read this section to the tune of `Let it go' from \emph{Frozen} 
\begin{enumerate}
\item Everyone can be a Bayesian, not just wildlife and fisheries biologists (aka HMM and state-space people)
\item Please, stop going on and on about priors. 
\item Conjugate priors as the crystal deodorant of priors (check Dan Simpson quote)
\item Let go of `random' versus fixed effects ideas
\item Let go of p-values and embrace numbers with units! (\href{https://www.youtube.com/watch?v=c3hxhv0lpI0}{`I am arbitrary but my story is often told ... '}
\end{enumerate}
\item Conclusions
\begin{enumerate}
\item Ecologists cannot simulate their stats (or simple systems for that matter). Evolutionary biologists can. (And the field is better for it.)
\item Maybe hint at that you need these skills (and unit testing) given rise of AI?
\end{enumerate}
\emph{Take home messages (maybe)}
\begin{enumerate}
\item You should not fit a model you cannot simulate
\item Fit simpler models
\item Know your nonidentifiability
\end{enumerate}
\end{enumerate}
\vspace{2ex}

Stuff missing a home in the outline .... 
\begin{enumerate}
\item `Best practices' workflow (or just `best practices'?)
\item  Process-based models versus theory versus workflow
\item $\alpha$ and $\beta$ power (Will writes this)
\item Community uncertainty and propagating uncertainty (Dietze)
\item Theoretical ecology is a more separated field in ecology compared to in evolution. 
\end{enumerate}


\emph{Things to decide/do}
\begin{enumerate}
\item Do we need an example? Cherries or the project with Heather?
\item Sample code (could get Will to do this) including ... 
\begin{enumerate}
\item Bad interaction
\item Bad prior
\item Including study and species (non-identifiability)
\end{enumerate}
\end{enumerate}


\section{Refs to check out...}

Very short keynote talk I gave. \\ % ~/Documents/git/teaching/stan/bayesianworkflows

Workflow in van der schoot (the paper with Ruth King?)\\
Gelman has a workflow paper \\

What Betancourt has.

Gabry has visualization in the workslow paper. 

\newpage
\section{My old tutorial!} % ~/Documents/git/teaching/bayesforever/misc
{\bf One Bayesian Workflow:} \\

Below I outline a basic Bayesian workflow for someone using \verb|Stan|. This is a short overview of a much deeper topic. See \href{https://www.nature.com/articles/s43586-020-00001-2}{Bayesian statistics and modelling} for a little more depth, including how to develop priors and your basic model formulation.\\

\emph{Note that you might need to start simple and build up to get all these steps to work, but this is my recommended approach to the basic workflow.}

\begin{enumerate}
\item Check your code. Write down your model -- I recommend as basic math, then write it in \verb|Stan| and simulate one set of test data (often in \verb|R|). I recommend you do this first because it will flesh out any issues in your simulated data (\verb|R|) code, which you need right away (but you don't need the \verb|Stan| code until step 3).
\begin{enumerate}
\item Write your \verb|Stan| code.
\item In \verb|R|, write simulated data where you {\bf write out all your parameters}, write $x$ and generate $y$ from your model. So, if I am doing simple regression ($y=mx+b$, with error $\epsilon$) then I would have to assign values to $m, b$ and $\epsilon$ and I would generate a vector of $x$ values, then I would simulate $y$ from those values.
\item Run your \verb|Stan| code on your simulated data. Check that your \verb|Stan| code returns your model parameters, if not, check your code, set your error lower and/or sample size higher. Keep checking until your \verb|Stan| output matches your parameters (note check your \verb|Stan| code and your simulated data code) and you trust both.
\end{enumerate}
\item Prior predictive checks: Check yo' priors.
\begin{enumerate}
\item Take your aforementioned \verb|R| code and set up priors for each parameter (e.g., distributions for $m, b, \epsilon$). {\bf In contrast to above, where you just set each parameter to one value, here you you want to draw multiple values for each parameter and visually check the output.} 
\item How do I check the output? Just like posterior predictive checks, this is up to you! At a minimum I recommend thinking of plots you will make with your model in the end (for publications) and plotting that given different prior values.
\item Your goal here is to check that your priors are reasonable, if they are not, adjust them. 
\item Unlike in Step 1, you do {\bf not} need to run \verb|Stan|---you have your parameters so you can just simulate data from them, and then plot, examine etc.. No \verb|Stan| at this step. 
\end{enumerate}
\item Now you can run your model on your real data!
\begin{enumerate}
\item Check the output, if you have divergent transitions, you need to re-parameterize your model (may tried a non-centered model or such).
\item I recommend ShinyStan here.
\end{enumerate}
\item Posterior predictive checks: How good or bad does your model do compared to your data?
\begin{enumerate}
\item Grab the parameter values from your fitted \verb|Stan| model. In posterior predictive checks \emph{these are the parameter values you use to simulate new data.}  
\item You can adapt your \verb|R| code for simulating data above, but use your estimated parameters from your fitted model. 
\item What should I look at? Ah, just like in prior predictive checks, you need to decide. Classic things are to look at the mean of 100 or so simulations of new data you generate versus your {\bf real} data. Also try the SD. Plot things! Look at the distribution. If you use the generated quantities block in \verb|Stan| ShinyStan will automatically generate a few plots but think hard about more.
\item When does this get hard? In hierarchical models (and other models with hyperparameters) as you have multiple levels you can generate---you can use \emph{your estimated parameters from your fitted model at all levels or generate your lower-level parameters} (for example, you can generate species means using your species $\mu$ and $\sigma$). You have to think about what you want your model to predict.
\item See how it is---do you see obvious problems caused by the distribution you selected or a grouping factor you're missing? If so, add it. And go back to Step 1. 
\item (No \verb|Stan| at this step.)
\end{enumerate}
\end{enumerate}

\end{document}
